\section[Problema 3]{Problema 3}

Nesse problema do caxeiro viajante, o Algoritmo Genético se destacou, apresentando os melhores resultados entre todos os demais algoritmos, enquanto todos os demais ficaram presos no mínimo local próximo de 19000, o GA chegou alcançar o mínimo local de 15538, ainda que seja um valor muito longe do mínimo global, que é de 6656, observamos que esse algoritmo ainda não havia chegado ao seu potencial máximo, pois os valores de novas populações ainda estavam oscilando muito, como pode ser visto na figura \ref{fig:problema-3-genetic-algorithm-funcao-objetivo-value}, se compararmos com as figuras dos problemas anteriores \ref{fig:problema-1-genetic-algorithm-funcao-objetivo-value} e \ref{fig:problema-2-genetic-algorithm-funcao-objetivo-value}, vemos claramente que mesmo após 100k iterações os valores atuais ainda não haviam convergido para um valor único, enquanto nos demais problemas essa convergência ocorreu pouco após a iteração 200. \\

O algoritmo genético apesar de possuir um peso computacional muito elevado, apresentou uma eficiência muito maior em encontrar novos mínimos locais, acreditamos que após algumas análises sobre esse problema que a aplicação de algumas heurísticas para melhorar a eficácia não só desse mas de todos os demais seja um caminho mais adequado, por exemplo evitando que path's cruzem com outros, verificamos que em geral a rota que segue o perímetro mais externo dos pontos, possui a distância mais curta. \\

O algoritmo Hill-Climbing apresentou boa performance para esse problema e conseguiu ao longo das iterações fugir de mínimos locais. \\

Os algoritmos Hill-Climbing com restart e Simulated Annealing apresentaram um comportamento muito parecido com números muito similares também.

\subsection{Algoritmo Hill-Climbing}

\begin{figure}[H]
\centering
\includegraphics[width=110mm]{imagens/otima/problema-3-hill-climbing-funcao-objetivo-best.png}
\caption{Dados da execução da função objetivo durante as 10 iterações.
\label{fig:problema-3-hill-climbing-funcao-objetivo}}
\end{figure}

\subsection{Algoritmo Hill-Climbing com Restart}

\begin{figure}[H]
\centering
  \begin{minipage}[b]{0.48\textwidth}
    \includegraphics[width=88mm]{imagens/otima/problema-3-hill-climbing-com-restart-funcao-objetivo-best.png}
    \caption{Dados da execução da função objetivo durante as 10 iterações por melhor valor.
    \label{fig:problema-3-hill-climbing-com-restart-funcao-objetivo-best}}
  \end{minipage}
  \hfill
  \begin{minipage}[b]{0.48\textwidth}
    \includegraphics[width=88mm]{imagens/otima/problema-3-hill-climbing-com-restart-funcao-objetivo-value.png}
    \caption{Dados da execução da função objetivo durante as 10 iterações por valor atual.
    \label{fig:problema-3-hill-climbing-com-restart-funcao-objetivo-value}}
  \end{minipage}
\end{figure}

\subsection{Simulated Annealing}

\begin{figure}[H]
\centering
  \begin{minipage}[b]{0.48\textwidth}
    \includegraphics[width=88mm]{imagens/otima/problema-3-simulated-annealing-funcao-objetivo-best.png}
    \caption{Dados da execução da função objetivo durante as 10 iterações por melhor valor.
    \label{fig:problema-3-simulated-annealing-funcao-objetivo-best}}
  \end{minipage}
  \hfill
  \begin{minipage}[b]{0.48\textwidth}
    \includegraphics[width=88mm]{imagens/otima/problema-3-simulated-annealing-funcao-objetivo-value.png}
    \caption{Dados da execução da função objetivo durante as 10 iterações por valor atual.
    \label{fig:problema-3-simulated-annealing-funcao-objetivo-value}}
  \end{minipage}
\end{figure}

\subsection{Algoritmo Genético}

\begin{figure}[H]
\centering
  \begin{minipage}[b]{0.48\textwidth}
    \includegraphics[width=88mm]{imagens/otima/problema-3-genetic-algorithm-funcao-objetivo-best.png}
    \caption{Dados da execução da função objetivo durante as 10 iterações por melhor valor.
    \label{fig:problema-3-genetic-algorithm-funcao-objetivo-best}}
  \end{minipage}
  \hfill
  \begin{minipage}[b]{0.48\textwidth}
    \includegraphics[width=88mm]{imagens/otima/problema-3-genetic-algorithm-funcao-objetivo-value.png}
    \caption{Dados da execução da função objetivo durante as 10 iterações por valor atual.
    \label{fig:problema-3-genetic-algorithm-funcao-objetivo-value}}
  \end{minipage}
\end{figure}